\documentclass{scrartcl}

\usepackage[utf8]{inputenc}
\usepackage[T1]{fontenc}

\usepackage{amsmath}
\usepackage{amssymb}

\usepackage[osf,sc]{mathpazo}
\usepackage[scaled=0.86]{berasans}
\usepackage[scaled=1.03]{inconsolata}

\begin{document}

\section{Simulating percolation}

There are two limits to approach the problem of percolation through a
lattice: in the limit of \emph{site percolation} two neighboring sites
on the lattice enable percolation.  Thus, the only information needed to
determine if a system is percolating is the \emph{lattice decoration},
i.e., the occupancies of all sites.  The second limit is \emph{bond
  percolation}, in which each connection between two neighboring sites
of the lattice can either be percolating or non-percolating.  

\subsection{Site percolation}

For an infinite lattice with a concentration $p$ of the conducting
species, the \emph{percolation probability} $P_{\infty}(p)$, i.e., the
probability that an occupied site is part of a percolating (infinite)
cluster is given by
%
\begin{align}
  P_{\infty}(p) 
  = \begin{cases}
    \hat{B}_{p}\,
    \Bigl( 
      \frac{p}{p_{\textup{c}}} - 1
    \Bigr)^{\beta_{p}}
    & \text{for}\quad p>p_{\textup{c}}
    \\
    0 & \text{else}
    \quad ,
  \end{cases}
  \label{eq:P_infty}
\end{align}
%
where the critical concentration $p_{\textup{c}}$ is the \emph{site
  percolation threshold}.  The prefactor $\hat{B}_{p}$ and the exponent
$\beta_{p}$ both depend on the lattice and the concentration.

\end{document}

%%% Local Variables: 
%%% mode: latex
%%% TeX-master: t
%%% End: 
